\documentclass[../main.tex]{subfiles}

\begin{document}
\begin{CJK*}{UTF8}{gbsn}

\section*{Exercice 1}

Soit $X_1, X_2, \cdots, X_n$ un échantillon de taille $n$ avec densité 
$f_{\theta}(x) = \frac{1 + \theta x}{2} \chi_{[-1,1]}(x)$, 
où $\theta \in [-1,1]$ est inconnu. Écrire $Y_i = \chi_{[0,1]}(X_i)$
pour tout $i \in \curly{ 1, 2, \cdots, n}$.
Estimer $\theta$ par la méthode des moments utilisant $\{X_i\}_{i=1}^n$ et 
calculer l'écart quadratique moyenne. 
Proposez un autre estimateur de $\theta$ qui améliore cet écart quadratique moyenne.
Répétez les mêmes tâches utilisant $\{Y_i\}_{i=1}^n$ et pas $\{X_i\}_{i=1}^n$.

\paragraph{Solution}
La méthode des moments consiste à égaler les moments théoriques aux moments empiriques calculés à partir de l'échantillon. Le premier moment (moyenne) d'une variable aléatoire $X$ avec densité $f(x | \theta)$ est donné par l'espérance mathématique $E[X]$.

Considérons la fonction de densité de $X$ donnée par 
\begin{equation}
f(x | \theta) = \frac{1 + \theta x}{2}, \quad \text{pour } x \in [-1, 1]
\end{equation}
L'espérance de $X$,$E[X]$, s'obtient en intégrant le produit de $X$ par la densité de probabilité sur l'intervalle de définition de $X$, ce qui nous donne :
\begin{equation}
E[X] = \int_{-1}^{1} x \cdot \frac{1 + \theta x}{2} \, dx
\end{equation}
Cette intégration aboutit à :
\begin{equation}
E[X] = \frac{\theta}{3} 
\end{equation}

\begin{equation}
E[X^2] = \frac{\theta}{4} 
\end{equation}

En utilisant la méthode des moments, nous égalisons le moment théorique, ici l'espérance calculée $E[X]$, avec le moment empirique, la moyenne d'échantillon $\bar{X}$, ce qui nous donne l'équation :
\begin{equation}
\bar{X} = \frac{\theta}{3}
\end{equation}
En résolvant cette équation pour $\theta$, nous obtenons l'estimateur de $\theta$ par la méthode des moments :
\begin{equation}
\hat{\theta}_{MM} = 3\bar{X}
\end{equation}

\begin{equation}
\hat{\theta}_{MM} = 3\left(\frac{1}{n}\sum_{i=1}^{n}X_i\right)
\end{equation}

L'Écart Quadratique Moyen (EQM) d'un estimateur $\hat{\theta}$ est défini par :
\begin{equation}
EQM(\hat{\theta}) = E[(\hat{\theta} - \theta)^2].
\end{equation}

\begin{equation}
E\left[\left(\frac{3}{n}\sum_{i=1}^{n}X_i - \theta\right)^2\right] = \frac{9}{n} E[X_1^2] + \frac{18}{n^2} \cdot \frac{n(n-1)}{2} \cdot E^2[X_1] + \theta^2 + 6\theta \cdot E[X_1]
\end{equation}

Nous procédons maintenant à la substitution de $E[X]$ et $E[X^2]$ dans notre expression précédente pour simplifier le calcul.

\begin{equation}
E\left[\left(\frac{3}{n}\sum_{i=1}^{n}X_i - \theta\right)^2\right] = \frac {\left(\frac{9}{2} + 8n \cdot \theta^2 - 2\theta^2\right)}{2n}
\end{equation}


\end{CJK*}
\end{document}
