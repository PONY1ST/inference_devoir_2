\documentclass[../main.tex]{subfiles}

\begin{document}
\begin{CJK*}{UTF8}{gbsn}

\section*{Exercice 1}

Soit $X_1, X_2, \cdots, X_n$ un échantillon de taille $n$ avec densité 
$f_{\theta}(x) = \frac{1 + \theta x}{2} \chi_{[-1,1]}(x)$, 
où $\theta \in [-1,1]$ est inconnu. Écrire $Y_i = \chi_{[0,1]}(X_i)$
pour tout $i \in \curly{ 1, 2, \cdots, n}$.
Estimer $\theta$ par la méthode des moments utilisant $\{X_i\}_{i=1}^n$ et 
calculer l'écart quadratique moyenne. 
Proposez un autre estimateur de $\theta$ qui améliore cet écart quadratique moyenne.
Répétez les mêmes tâches utilisant $\{Y_i\}_{i=1}^n$ et pas $\{X_i\}_{i=1}^n$.

\paragraph{Solution}

Écrire $\hat{\theta_x}$ et $\hat{\theta_y}$ 
pour les estimateurs de $\theta$ utilisant la méthode de moments 
basés sur $\{X_i\}_{i=1}^n$ et $\{Y_i\}_{i=1}^n$ respectivement.
Comme $\mathbb{E}[X_1] = \frac{\theta}{3}$
et $\mathbb{E}[Y_1] = \frac{1}{2} + \frac{\theta}{4}$,
on a $\hat{\theta_x} = \frac{3}{n}\sum_{i=1}^{n}X_i$, $\hat{\theta_y} = \frac{4}{n}\sum_{i=1}^{n}Y_i - 2$.
Alors :

\begin{equation*}
    \begin{split}
    &\tab[0.4cm] \mathbb{E} \squared{(\hat{\theta}_x-\theta)^2} = 
    \mathbb{E} \squared{\frac{9}{n^2} \sum_{i=1}^n X_i^2 + 
    \frac{9\times 2}{n^2} \sum_{i=1}^n \sum_{j=1}^{i-1} X_i X_j + \theta^2 - \frac{6\theta }{n} \sum_{i=1}^n X_i}
    \\ & = \frac{9}{n} \frac{1}{3} + \frac{18}{n^2} \frac{n(n-1)}{2} 
    \bracket{\frac{\theta}{3}}^2 + \theta^2 - 6 \theta \frac{\theta}{3} = \frac{3}{n} - \frac{\theta^2}{n}
    \end{split}
\end{equation*}

Ici on utilise le fait que $\mathbb{E}[X_1^2] = \frac{1}{3}$. Ensuite :

\begin{equation*}
    \begin{split}
    &\tab[0.4cm] \mathbb{E} \squared{(\hat{\theta}_y-\theta)^2} = 
    \mathbb{E} \squared{\frac{16}{n^2} \sum_{i=1}^n Y_i^2 + 
    \frac{16\times 2}{n^2} \sum_{i=1}^n \sum_{j=1}^{i-1} Y_i Y_j + \theta^2 + 4 - \frac{16}{n} \sum_{i=1}^n Y_i -  \frac{8\theta}{n} \sum_{i=1}^n Y_i + 4 \theta}
    \\ & = 
    \frac{16}{n}\bracket{\frac{1}{2} + \frac{\theta}{4}} + \frac{32}{n^2} \frac{n(n-1)}{2} \bracket{\frac{1}{2} + \frac{\theta}{4}}^2
    + \theta^2 + 4 - 16 \bracket{\frac{1}{2} + \frac{\theta}{4}} -  8\theta \bracket{\frac{1}{2} + \frac{\theta}{4}} + 4 \theta
    = \frac{4}{n} - \frac{\theta^2}{n}
    \end{split}
\end{equation*}

Pour améliorer les écarts quadratique moyen, on propose des nouveaux estimateurs $\hat{\theta}_{x}^{adj}$
et $\hat{\theta}_{y}^{adj}$ comme suit :

\begin{equation*}
    \hat{\theta}_{x}^{adj} = \bracket{\frac{3}{n}\sum_{i=1}^{n}X_i}X_i \chi_{[-1,1]}\bracket{\frac{3}{n}\sum_{i=1}^{n}X_i} 
    + \chi_{(1,\infty)}\bracket{\frac{3}{n}\sum_{i=1}^{n}X_i} - \chi_{(-\infty,-1)}\bracket{\frac{3}{n}\sum_{i=1}^{n}X_i}
\end{equation*}
\begin{equation*}
    \hat{\theta}_{y}^{adj} = 
    \bracket{\frac{4}{n}\sum_{i=1}^{n}Y_i - 2} \chi_{[-1,1]}\bracket{\frac{4}{n}\sum_{i=1}^{n}Y_i - 2} 
    + \chi_{(1,\infty)}\bracket{\frac{4}{n}\sum_{i=1}^{n}Y_i - 2} - \chi_{(-\infty,-1)}\bracket{\frac{4}{n}\sum_{i=1}^{n}Y_i - 2}
\end{equation*}

Alors :

\begin{equation*}
    \begin{split}
        &\tab[0.4cm]
    \mathbb{E} \squared{(\hat{\theta}_{x}^{adj}-\theta)^2} =
    \mathbb{E} \squared{(\hat{\theta}_{x}-\theta)^2 ; \curly{\hat{\theta}_{x} \in [-1,1]}} + 
    \mathbb{E} \squared{(1-\theta)^2 ; \curly{\hat{\theta}_{x} > 1}} + 
    \mathbb{E} \squared{(-1-\theta)^2 ; \curly{\hat{\theta}_{x} > 1}}
    \\ & < \mathbb{E} \squared{(\hat{\theta}_{x}-\theta)^2 ; \curly{\hat{\theta}_{x} \in [-1,1]}} + 
    \mathbb{E} \squared{(\hat{\theta}_{x}-\theta)^2 ; \curly{\hat{\theta}_{x} > 1}} + 
    \mathbb{E} \squared{(\hat{\theta}_{x}-\theta)^2 ; \curly{\hat{\theta}_{x} > 1}} = \mathbb{E} \squared{(\hat{\theta}_{x}-\theta)^2}
    \end{split}
\end{equation*}

On souligne que l'inégalité est stricte parce que les ensembles 
$\curly{\hat{\theta}_{x} > 1}$ et $\curly{\hat{\theta}_{x} < -1}$ ont mesures positives.
Alors, $\hat{\theta}_{x}^{adj}$ améliore l'écart quadratique moyen de $\hat{\theta}_{x}$.
Changer tout $x$ à $y$, on montre que $\hat{\theta}_{y}^{adj}$ améliore l'écart quadratique moyen de $\hat{\theta}_{y}$.
Les démonstrations sont complètes. ////

\end{CJK*}
\end{document}
