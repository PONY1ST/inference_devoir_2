\documentclass[../main.tex]{subfiles}

\begin{document}
\begin{CJK*}{UTF8}{gbsn}

\section*{Exercice 3}

Soit $g : \mathbb{R} \to \mathbb{R}$ une fonction continue et décroissante. 
On suppose que toutes les fonctions $f_{\theta} : \mathbb{R} \to \mathbb{R}$, 
où $\theta \in \mathbb{R}$, définies par $f_{\theta}(x) = g(\abs{x - \theta})$
sont densités de probabilité.
Soit $x_1, x_2, \cdots, x_n$ un échantillon de taille $n$ avec densité
$f_{\theta}$, où $\theta \in \mathbb{R}$ est inconnu.
Écrire $x_{(1)} \leqslant \cdots \leqslant x_{(n)}$ pour les statistiques d'ordre.
Montrez que l'estimateur de maximum de vraisemblance de $\theta$ existe et se trouve 
dans l'intervalle $[x_{(1)}, x_{(n)}]$.

\paragraph{Solution}

à faire
\end{CJK*}
\end{document}