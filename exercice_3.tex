\documentclass[../main.tex]{subfiles}

\begin{document}
\begin{CJK*}{UTF8}{gbsn}

\section*{Exercice 3}

Soit $g : \mathbb{R} \to \mathbb{R}$ une fonction continue et décroissante. 
On suppose que toutes les fonctions $f_{\theta} : \mathbb{R} \to \mathbb{R}$, 
où $\theta \in \mathbb{R}$, définies par $f_{\theta}(x) = g(\abs{x - \theta})$
sont densités de probabilité.
Soit $x_1, x_2, \cdots, x_n$ un échantillon de taille $n$ avec densité
$f_{\theta}$, où $\theta \in \mathbb{R}$ est inconnu.
Écrire $x_{(1)} \leqslant \cdots \leqslant x_{(n)}$ pour les statistiques d'ordre.
Montrez que l'estimateur de maximum de vraisemblance de $\theta$ existe et se trouve 
dans l'intervalle $[x_{(1)}, x_{(n)}]$.

\paragraph{Solution}
La fonction de vraisemblance est définie par
$L(\theta) = \prod_{i=1}^{n} g(|x_i - \theta|)$.

Premièrement, examinons la continuité de $L(\theta)$. Étant donné que $g$ est continue et décroissante,
pour tout $x_i$ et $\theta$, la fonction $g(|x_i - \theta|)$ est également continue par rapport à $\theta$.
Par conséquent, comme $L(\theta)$ est le produit des fonctions continues $g(|x_i - \theta|)$ pour $i$
allant de 1 à $n$, $L(\theta)$ est elle-même continue par rapport à $\theta$.

Considérer $\theta$ dans l'intervalle compact $[x_{(1)}, x_{(n)}]$ est justifié par le fait 
que toutes les observations $x_i$ se situent dans cet intervalle. La décroissance de la 
fonction $g$ implique que la valeur de $L(\theta)$ diminue hors de cet intervalle, 
rendant l'estimation de $\theta$ optimale uniquement à l'intérieur de $[x_{(1)}, x_{(n)}]$.

La continuité de $L(\theta)$ sur cet intervalle compact assure, selon le théorème des 
valeurs extrêmes, que $L(\theta)$ atteint son maximum dans $[x_{(1)}, x_{(n)}]$. 
Cela confirme l'existence de l'estimateur du maximum de vraisemblance dans cet intervalle,
exploitant ainsi directement les propriétés des ensembles compacts.

En conclusion, la combinaison de la continuité de $L(\theta)$ et des propriétés de l'intervalle compact
$[x_{(1)}, x_{(n)}]$ permet d'affirmer que l'estimation du maximum de vraisemblance existe et est
confinée à l'intérieur de cet intervalle. Cela illustre l'importance de la continuité et des intervalles
compacts dans la détermination de l'estimation du maximum de vraisemblance, tout en mettant en évidence
la propriété de maximisation de la fonction de vraisemblance dans l'intervalle donné.

\end{CJK*}
\end{document}
