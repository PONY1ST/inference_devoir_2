\documentclass[../main.tex]{subfiles}

\begin{document}
\begin{CJK*}{UTF8}{gbsn}

\section*{Exercice 3}

Soit $g : \mathbb{R} \to \mathbb{R}$ une fonction continue et décroissante. 
On suppose que toutes les fonctions $f_{\theta} : \mathbb{R} \to \mathbb{R}$, 
où $\theta \in \mathbb{R}$, définies par $f_{\theta}(x) = g(\abs{x - \theta})$
sont densités de probabilité.
Soit $x_1, x_2, \cdots, x_n$ un échantillon de taille $n$ avec densité
$f_{\theta}$, où $\theta \in \mathbb{R}$ est inconnu.
Écrire $x_{(1)} \leqslant \cdots \leqslant x_{(n)}$ pour les statistiques d'ordre.
Montrez que l'estimateur de maximum de vraisemblance de $\theta$ existe et se trouve 
dans l'intervalle $[x_{(1)}, x_{(n)}]$.

\paragraph{Solution}
L'estimateur du maximum de vraisemblance $\theta_{\text{MLE}}$ est la valeur de $\theta$ qui 
maximise la fonction de vraisemblance $L(\theta) = \prod_{i=1}^{n} g(|x_i - \theta|)$, avec 
$g$ étant une fonction continue et décroissante. La propriété de décroissance de $g$ entraîne 
que $L(\theta)$ est maximisée lorsque $\theta$ se situe au plus proche des $x_i$, et en 
considérant l'ensemble des observations dans l'intervalle compact $[x_{(1)}, x_{(n)}]$, 
$L(\theta)$ atteindra nécessairement son maximum dans cet intervalle d'après le théorème 
des valeurs extrêmes. Par conséquent, $\theta_{\text{MLE}}$ doit exister à l'intérieur de 
$[x_{(1)}, x_{(n)}]$.

\end{CJK*}
\end{document}
