\documentclass[../main.tex]{subfiles}

\begin{document}
\begin{CJK*}{UTF8}{gbsn}

\section*{Exercice 2}

On a un échantillon de taille $2$ de $\text{Cauchy}(\theta, 1)$, où le centre $\theta \in \mathbb{R}$ est inconnu.
Estimer $\theta$ selon la méthode du maximum de vraisemblance. 

\paragraph{Solution}
La fonction de vraisemblance pour deux variables aléatoires indépendantes $X_1, X_2$ est :
\begin{equation*}
L(\theta | x_1, x_2) = f(x_1 | \theta) \cdot f(x_2 | \theta) = \left( \frac{1}{\pi} \frac{1}{1 + (x_1 - \theta)^2} \right) \left( \frac{1}{\pi} \frac{1}{1 + (x_2 - \theta)^2} \right)
\end{equation*}

Introduisons le nouveau paramètre $\eta = \theta - \frac{x_1 + x_2}{2}$ et $y = \frac{x_1}{2} - \frac{x_2}{2} $, ce qui implique :

En remplaçant $\theta$ par $\eta$ dans la fonction de vraisemblance :
\begin{equation*}
L(\eta) =\frac{1}{\pi^2} \cdot \frac{1}{1 + (y - \eta)^2} \cdot \frac{1}{1 + (y + \eta)^2} = \frac{1}{\pi^2 \left[(\eta^2 + 1 - y^2)^2 + 4y^2\right]}
\end{equation*}
Trouver la valeur minimale de \(\eta^2 + 1 - y^2\) permet de déterminer la valeur maximale de \(L(\eta)\).
On conclut que lorsque \(\eta = \pm \sqrt{y^2 - 1}\), \(L(\theta)\) atteint sa valeur maximale.
Donc on peut calculer l’extimateur de $\theta$, $\hat{\theta}$ :
\begin{equation*}
\hat{\theta} = \pm \sqrt{\left[\frac{(x_1 - x_2)^2 - 4}{4}\right]} + \frac{x_1 + x_2}{2}
\end{equation*}

\end{CJK*}
\end{document}
