\documentclass[../main.tex]{subfiles}

\begin{document}
\begin{CJK*}{UTF8}{gbsn}

\section*{Exercice 2}

On a un échantillon de taille $2$ de $\text{Cauchy}(\theta, 1)$, où le centre $\theta \in \mathbb{R}$ est inconnu.
Estimer $\theta$ selon la méthode du maximum de vraisemblance. 

\paragraph{Solution}
Étant observé $x_1$ et $x_2$,
la fonction de vraisemblance $L : \mathbb{R} \to \mathbb{R}$ définie par, pour tout $\theta \in \mathbb{R}$ :

\begin{equation*}
L(\theta) = \frac{1}{\pi^2} \frac{1}{1 + (x_1 - \theta)^2} \frac{1}{1 + (x_2 - \theta)^2} 
\end{equation*}

Définir $\eta = \theta - \frac{x_1 + x_2}{2}$ et $y = \frac{x_1}{2} - \frac{x_2}{2} $.
Il s'agit de minimiser le dénominateur parmi tout $\eta \in \mathbb{R}$, qui est :

\begin{equation*}
    \pi^2 \squared{(\eta^2 + 1 - y^2)^2 - (1-y^2)^2 + (y^2+1)^2}^2
\end{equation*}

Si $y^2 \geqslant 1$, on voit immédiatement que $\eta = \pm \sqrt{y^2 - 1}$ est le minimum global.
Si $y^2 < 1$, le développement de dénominateur est un polynôme de $\eta^2$ avec des coefficients positifs
et $\eta = 0$ est le minimum global. Alors, 
l'estimateur de maximum de vraisemblance de $\theta$, $\hat{\theta}$, est :

\begin{equation*}
    \hat{\theta} = 
    \begin{cases}
        \frac{x_1 + x_2}{2} \tab[10.1cm] \bracket{\frac{x_1-x_2}{2}}^2 \leqslant 1 \\
        \sqrt{\squared{\frac{x_1-x_2}{2}}^2 - 1} + \frac{x_1 + x_2}{2} 
        \text{ou} -\sqrt{\squared{\frac{x_1-x_2}{2}}^2 - 1} + \frac{x_1 + x_2}{2} \tab[2cm] \bracket{\frac{x_1-x_2}{2}}^2 > 1
    \end{cases}
\end{equation*}

Le calcul est complet. ////

\end{CJK*}
\end{document}
